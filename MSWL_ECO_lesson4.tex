\chapter{FLOSS business models}

\section{Introduction}\label{lesson-4-introduction}

FLOSS 

The one million dollar question \emph{Which model has success.}

Different FLOSS business models inside the market. Those sample are from two years ago, so don't take any strict conclusions because this models are continusly evolving.

\subsection{Origin}\label{lesson-4-origin}

Open innovation rise business models.

\subparagraph{Main Content}

Assets of the company to make money, culture, employees and knowledge as a part
of the capital that we will put aside money.

Open innovation stares at knowlegment and its use.

We begin with the example of a company where its product is the reference and
not the knowledge of its employees to manufacture the product, this case would
be:

If only produces the product strictly like a dictate of the company , although
exists an employee who knows how to manufacture the product more better than the
company its self therefore is underused, the resource is maximized.

Unlike the knowledge unification among employees so as to find the best way for
developing the product in which they all participate and the best improvments or
suggestions tending to envolve are welcomed as nothing is permanent.

How companies manage the knowledge ? \emph{The knowledge is essential for the
company and therefore it has to protect}, this is against with the
principles of  FLOSS, so as the competion does not compete with their
knowledge. The private treatment of the knowledge restricts the radius of
improvement over the different cases in which the knowledge is being shared because by sharing knowledge they both
win. The rapid innovation against the protection of the company - of the
enviroment where closes to outside opinions.

The knowledge exchange begins between employees by companies
as for example in case of Silicon Valley at the middle of 70's where the
companies needed engineers and an engineer's life in a company was no more than
18 months so the knowledge was flowing between all companies until convincing a critical mass
knowledge exchange for the transfer of employees.

\begin{itemize}
  \item Case Study: \emph{Gold Corp} Predictive model.
  \item Case Study: \emph{Netflix}
\end{itemize}

Predictive models developed by people outside of the company rewarding the ``winning model''

Beginning with NDAs (non-disclosure agreement) as a point in Software
companies,they delimit communication and dissemination of knowledge. 

Until the gap came in where people believed that innovation should not be the
solely subject to the company and that in this way it would grow faster, the
feedback and the contact with 'competition'.

This model could be extrapolated to any other business, for example the
inclusion of customer in the design process to receive feedback from the user community.

Kickstarter:Leaving the market foresight to the customers before the prouduct is
being manufactured

\section{Current trends}\label{lesson-4-current-trends}

\emph{"There is no silver bullet"}

Be attentive to the market , not only to which one belongs but to external
markets so that can be defined as a specialization or an expertizing of the
market.It rewards collaboration with users generating knowledge on par with innovation
so that the interested user could improve the product.

\section{False myths}\label{lesson-4-false-myths}

The sales price is the price which people are able to pay , and it has nothing
to do with the production cost. 
The value that is given by user/client can be evaluated : 


\begin{itemize}
    \item The value of using the tool , investment.
    Savings of 5000 euros  for one computer but you pay 500 Euros as a sale
    value.
    \item The sales value.
\end{itemize}


The 75\% of the programs is dedicated to the maintenance while the rest is
dedicated to the innovation.
The value of software is not measured by the \emph{value of replacement}. 

It doesn't mind the cost of the production but what it would cost if you don't
having to produce the product again, associated with the physical assets.


Expected value of the future service.

The upper limit in the software is marked by future service that offers to me ,
by looking forward in what he has not been done so far.

Software stability , support, documentation , adaption or how can you get a
certification.Unlike the physical goods you don't think as soon as you can find
the same product as they can get the same product updated while that in a
physical if you look back.

    Services == Source of revenues

In Software you don't purchase the product , you evaluate the service that product offers.

The organization of Free Software model allows offering best services:
\begin{itemize}
    \item Users who become clients.
    \item Scalability in failure identification. Successful programs use to have less errors, where success measured in user.
    \item Share risks and production costs. Development cost is shared among the actors.
    \item Monopolistic practices are very difficult. A fork allows to escape from product monopolization.
\end{itemize}

\emph{If you don't have users , you don't have any available business model.}
%\end{Main Content}

\section{Where is FLOSS ?}\label{sec:floss-bm}

FLOSS by itself is not, and it has never been, a business model.

\subsection{A guide for SMEs}

Mobiles: sell mobiles as a business model using FLOSS, more viability, cost
 structure, not the model itself.

If the company

Si la empresa no utilizara FLOSS no sería una empresa viable, es decir, sin el
 uso de FLOSS la empresa no existiría.

\subsection{FLOSSMETRICS}

Study from 2008/2009 analysing 218 companies that 25\% of their total revenues
 directly or indirectly from FLOSS.

Identifying common business strategies around FLOSS.

\subsubsection{3 axes}

\begin{itemize}
    \item Software model
    \begin{itemize}
        \item Propietary vs libre software
    \end{itemize}
    
    \item Development model
    \begin{itemize}
        \item Barries to collaboration
        \item Single developer/reducted group vs. large communty, global outrech.
    \end{itemize}
    
    \item Business model
    \begin{itemize}
        \item Type of revenues model.
        \item Numerous options: Training, support, on-demand changes, productizing, SaaS, etc.
    \end{itemize}
\end{itemize}

Development model
Google Android - Private production, central model.
GCC - Libre, low barriers, medium group.
GLibC - Libre, low barriers, little group.

Business model
SaaS using private model.

\begin{comment}
\section{Strategic uses of Libre Software}

\emph{No cambias el modelo de negocio tradicional pero te apoyas en productos FLOSS para reforzalo.}

Es muy difícil encontrar una empresa que no utilice ningún producto de FLOSS. FLOSS es adaptativo
para los requisitos de una empresa. Traducciones, adaptaciones a distintos tipos de hardware,
interoperabilidad, entornos organizativos, culturales, nuevas tecnologías.

Otra tendencia, es la estrategia de liberar distintas distribuciones de software como FLOSS
para crear una comunidad libre alrededor.

Apple, Microsoft, Facebook, liberan partes de su código para crear una comunidad para mantener
el producto por parte de la misma.

Desequilibrios.

Usuarios que no quieren pagar por el Software privativo y se conforman con el programa libre y gratuito.
Pueden llegar a mejorar al modelo privativo. El privativo se ve desafiado cuando aparece una solución FLOSS
fiable que genere una aceptación en el mercado. (sitemas embebidos)

\section{Carlo Daffara taxonomy}

\begin{itemize}
    \item Dual licencing: FLOSS version and propietary version. MySQL Enterprise and MySQL FLOSS.
    \item Open core: Allows mixing FLOSS and proprietary elements. MSExcange mail model.
    \item Product specialists: Superior knowledge, additional services. Best service for the product.
    \item Platform providers: Integration, product testing. PaaS: FLOSS producto integration, glue products. Test by your own.
    \item Aggregate support providers: Primer nivel de soporte para diferentes tipos de software libre. Escalando el soporte entre todos los niveles.
    \item Selection/consulting companies: Closer to the analyst role, minimum impact on FLOSS communities.
    \item Legal certification and consulting : Assessment on license compatibility. Certificate interoperability.
    \item Training and documentation: Either as part of a broader support contract or companies exclusively devoted to this market area. User guides, documentation, courses.
    \item R&D cost sharing : Initial investment + creating community to reduce R&D costs. Create a community to reduce I+D cost % haciendo que la gente aporte desde fuera para mantenerse al día mediante la herramienta de la comunidad. OpenStack from NASA throught RackSpace sharing development cost.
    \item Indirect revenues: Baseline for sales of associated products or services (commodities).
\end{itemize}

\subsection{Dual (or multiple) licensing)}

Tienes que ser el dueño de todo el código, los contribuyentes ceden sus derechos de copyright para
licenciar el producto. El simple hecho de tener que firmar un documento echa hacia atrás las contribuciones de la comunidad.

MySQL integrado en otros productos para que el producto no se liberase siendo privado y de esta forma no violar la licencia GPL.
Relacionado con la licencia, la licencia Apache y privativo no hubiera funcionado ya que Apache permite el uso de su código dentro
de un producto privado.

\subsection{Open core}

Los ingresos o beneficios se centran en los componentes privativos alrededor 
del núcleo libre siendo esa misma empresa la única que puede desarrollar 
las mejoras de sus servicios. Todo el núcleo es libre.

Weakness: Desarrollar el servicio privativo mediante FLOSS, replicando sus
funcionalidades, reimplementación de un producto popular.

Suelen tener licencias duales para poder proporcionar la versión FLOSS y la privada.

Base grande de usuario, visibilidad y oportunidad de convertirlos a usuarios premium para 
obtener los beneficios.

Ganan valor de marca al ser reconocidos por la liberación del core.

Ejemplo: Implementación de Apache SSL. 
Valor añadido al software al ser la compañía la propietaria del núcleo.

Es un modelo de negocio libre controvertido ya que genera componentes privatidos a 
través de su plataforma libre, el núcleo.

Hay un compenente de duda alrededor de estos proyectos ya que no se abren a la
 comunidad al completo al desarrollar recursos privativos para la propia empresa.
 
Otra forma más positiva de verlo, es que atrae a diferentes empresas para 
invertir en el producto al poder desarrollar módulos privados. La empresa gasta
 un gran esfuerzo en la viabilidad del núcleo ya que si no se extiende sus 
 servicios no tendrían un nicho de negocio.

¿ Gana más recursos ? o se puede ver perjudicada la empresa por este 
comportamiento del modelo de negocio. ¿ De que depende que le convenga a una 
empresa adoptar este modelo ? 

Es un modelo muy cercano al privativo que intenta de alguna manera mantener al 
cliente cercado e inmóbil. Este modelo puede tender a desaparecer porque se puede
dar el caso que otra comunidad, proyecto o empresa que avance más rápido que 
la solución ofrecida por la empresa y por lo tanto se vea obligada a liberar el 
código, siendo demasiado tarde.

El modelo se mueve a través de una ventana temporal, es decir el 
openCore no parece viable de por vida, tiende a la liberación completa 
del código. Si el producto tiene éxito, es mucho más fácil que ocurra.

SendMail, SugarCRM.

\subsection{Product specialists}

Tengo el mejor conocimiento del producto certificado.

LPI y RedHat: dos certifiaciones alrededor de Linux. 

Modelo de Redhat en el campo del could computing: Incluir desarrolladores en el 
grupo de OpenStack y convertirse en una empresa especialista, no hace falta ser el creador del producto.

Weakness: Que otra empresa con mejor imagen, más nombre, mejores desarrollos desborde tu marca conseguida
 y por lo tanto quedes relegado a un segundo plano.
 
\subsection{Platform providers}

\begin{itemize}
  \item Selection, support, integration and services around a set of projects integrated in a single, tested and verified product.
  \item Key points; 
  \begin{itemize}
    \item Verification services
    \item Additional services    
  \end{itemize}
  \item Copyright ownership prevents direct copy (not cloning).
  \item Example: Redhat.
  \item The source code is libre software, but the product name and logo are trademarks.
  \item Considerable effort to eliminate them from files, doocumentation, etc.
\end{itemize}

Modelo de negocio de integración: tabletas de marca blanca, software, hardware, etc\ldots
Reproductores multimedia 'carrefour'.

Especialización y conversión en marca blanca de una tecnología.

\subsection{Aggregate support providers}

Parecido al soporte clásico pero con la diferencia de que es a todos los niveles.

Dos caminos a elegir:
Mediante el uso de recursos propios o la contratación de terceros para la resolución de los problemas.
Frontdesk; gestiona las peticiones a través de las empresas que conozcan el producto.

Clear benefit for large projects whose costs could raise due to excessive diversification of support channels (comprehensive help-desl).

OpenLogic: experts in Linux.

\subsection{Selection/consulting companies}

No desarrollan software, se encargan de proveer soluciones asesorando al cliente. Pruebas, análisis de rendimientos, conocimientos del territorio.

Gente experta en un campo dedicado a hacer análisis de un producto en cuestión. 
Estas empresas proporcionan herramientas que ayudan a reproducir resultados para interpretarlos haciendo análisis más completos.

Open WebApps

\subsection{Legal certification/consulting}

Palamida, Blackduck (mediante ohloh), Sonatype.

Especialización en las licencias de software y análisis de código para buscar posibles incompatibilidades.

\subsection{Training and documentation}

Training and documentation not certifiation. GBDirect. 

\subsection{R&D Const sharing}

Memo (nokia), nokia sólo desarrollaría una parte y el resto se quedaría en manos de la comunidad. Este modelo
 es previo a Android. Sabían del coste de software pero no eran un empresa productora, se dedicaban a la 
 fabricación de los terminales por lo que optaron por este modelo, dejar que el software creciera de mano
 de la comunidad. KDE, QT, Linux eran partes que formaban el núcleo de Memo.
 
OpenStack: RackSpace se dedica al mercado del cloud, vende clouds. Productizó OpenStack compartiendo el gasto 
del desarrollo con la comunidad (HP, Redhat, Intel\ldots). No es su nicho de mercado y de esta forma el desarollo
 de software libre lo comparte con los demás para aprovecharse del producto.

\subsection{Indirect revenues}

Intel, Dell, O'Relly. Perl fue el primer caso de la financiación de un proyecto de software libre a Larry Wall 
para que éste escribiese un libro para su editorial manteniendo los gastos del proyecto a su costa para sacar 
la próxima versión de Perl.

% Include graphic.

Open source going mainstream - Gartner group report.

Se duplican los beneficios cada tres años después del crecimiento del uso de software libre.

\ldots
\end{comment}

\section{Conclusions}\label{conclusions}
