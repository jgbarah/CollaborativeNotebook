\chapter{Introduction and motivation}

\section{Introduction to Economic Aspects and FLOSS}\label{ECO_FLOSS} 
The first assumption around FLOSS is how particular the FLOSS environment is. For this reason, business environment is also particular, and it has to be considered that means to make business have to be different from privative software business model, at least, at the beginning.\\
\\
\textbf{Privative software model}: It is based, basically, in two main activities:
- Licenses: Based on payment around a license that allows software programs usage.Normally, code is protected.
- Services: Support and other services around software. An activity that can be imitate on FLOSS software business model to make profit.\\
\\
This subject is focused, on the one hand, on how to make a profit business taking into account freedom of FLOSS.
But, on the other hand, it focuses on other aspects, as, for example:\\ 
\textbf{Is free software model "sustainable"?} For example: Will Libre Office competitive against Microsoft Office?\\
\\
Sustainability is very important for Free Software Projects to maintain alive in time, and this issue is not related to business around a particular FLOSS project. Both concepts are not exactly the same, although they are interrelated. If business models exists, sustainability over time is more probable\\
\\
There is an important issue to be resolved on facing business around Free Software: \textbf{identifying market niche}. In some particular cases, Free Software is not a good option, but the only option. Would Linux be today the same folowing a privative development model?\\
\\
Related to previous assumption, it has to be clarified that Free Software is not exactly altruism. Sometimes, Free Software is the best option in business. 10 years ago, no considerations on Free Software was considered in companies. Nowadays, thorough considerations are taken, considering Free Software strategies on companies global business plans.\\
\\
Another Consideration will be carried out. It has to be differentiated which companies produce software against companies that are only Free Software consumers. Business models differ taking into account this issue as well.\\
\\
Other times, FLOSS is considered as the \textbf{best tool} to achieve a particular target. Free Software, for this reason,  is used as the best way to achieve a company goal. One example is Android Operating System. Android may not be generating big profits for Google, but Google, as company, accomplished the target to reach a position in the market that allows them not to depend on Apple or Microsoft Mobile Operating Systems.\\
\\
Another concept around Free Software is \textbf{Open Innovation}. This concept is very wide, and applies to innovation in very different fields, from Sports to Fashion. Open Innovation is interrelated to FLOSS, as it takes concepts and procedures from FLOSS projects environment.\\
\\
FLOSS Market is not an insignificant Market. To clarify some numbers around FLOSS economy, Foundation incomes can be studied.Some good examples around bids of Foundations:\\
- Mozilla Foundation: around 100 Million Dollars\\
- Open Stack: around 10 Million Dollars\\
Apart from foundations, some enterprises are earning money. Red Hat, for example, is achieving profits in the last years. In 2011, total revenue was \$909.3 million, an increase of 22\% over the prior year, and subscription revenue was \$773.4 million, up 21\% year-over-year. 
\\
\section{Market Numbers around FLOSS}\label{MARKET_FLOSS} 
Some numbers can be analyzed around FLOSS. A good entry point is netcraft. This company analyzes Market based on Web Servers queries. They provide interesting reports, very trustable, as they are crawling machines to get the information, so the way the information is extracted is very different to more traditional reports based on surveys or other mechanisms.\\
\\
Related to Web Servers Market report analysis, it can be asserted that, in some cases, the market is leaded by a FLOSS Project (Apache in this case). Apache could get enough resources to keep being the leaders across the time. Moreover this, there are companies and individuals that make big effort on Apache. Companies as IBM, for example, did not try to create a web server, but instead they bet for Apache, as they do not want its competitors to gain the market. Apache has reached to be a standard de facto, so to sell a Web Server different from Apache, you can not sell anything the same or with less performance than Apache, especially when it is free.\\
\\
Around Browsers Market, however, other curious paradigm is happening: Webkit is base for Safari and Android Browser, and both are the leaders for Mobile Browsers. Curious as Google and Apple fight each other in many other businesses.\\
\\
Analyzing Market Reports some basic concepts can be foregrounded: 
- \textbf{Unstable Market}: leader having less than 60\% of the Market (Web Browsers Market, for example is an unstable Market).\\
\\
On the other hand, some other considerations arise around particular movements made by Companies around Software business, and how they change Market. Apple, for example,  hanged the Market when it allowed application to be sold with no exclusive rights, but with a percentage (30\%, a big amount taking into account the number of applications sold). On the other hand, Google introduced a wider and opener strategy, with a Free Operating System too. They are both examples on how selection of FLOSS strategy is important, as nobody thought Android could fight against Apple.\\
\\
Another interesting report shows that, in year 2008, many enterprises (around 40\%) were not using free software, but were interested on it. Other 30-35\% were not interested. Nowadays, everybody having a big/medium company knows that FLOSS is key business for their IT models.\\
\\
Another report related to Big computers market was shown. The main question around Super Computer Market, nowadays, is not who uses Linux, but wich distribution of Linux use each Super Computer. Linux Kernel is the undisputed leader in the market and, however, it is also a key player in many other different markets, as tMobile Devices. That is really something new in informatics, and has to do with openness, adaptability, etc. It is curious that, in Desktop, Linux never ends up entering the Market, when it is an example of a nice Operating System running on very different systems.\\
\\
To summarize, it has to be stated that FLOSS is a particular case, in technology and, of course, in economics.\\\\
It also has to be considered that, Internet is changing the model. First in time, things were done by small groups in small towns. Later, things were done by Big Enterprises or States. Nowadays, we are goint back to the "small town times", but this time small towns are interconected, as Internet is used to synchronise people. It also has to be clarified that,  sometimes, towns are not so small, as in the Wikipedia case.\\
\\
Economical traditional theories count on consumers on producers, and resources that one generates, and others consume. Market Theory, Supply and Demand, Value Chain, etc. In FLOSS somethings happen in the same format. For now Linux to be widely used, Linux had to be started 20 years ago. The difference is that traditional markets rules are not appliable:\\
- First of all, there is no price. With no price, 90\% of theories are not valid. 
- Apart from that, a company paying money to a FLOSS project (i.e.:Mozilla Foundation) can not valuate how much money will recover or win, at least directly.\\ 
- But another question arises:In fact, is it relevant? Is relevant to know how much money will a Company pay to acquire Mozilla? Market theories can not explain all the models and cooperation ways. There are economist that study and specialized in this paradigm.\\
\\

\section{A new paradigm in internet}\label{MARKET_FLOSS} 

A video was played in class to demonstrate the new paradigm around Internet. The most important conclussions after watching are:\\
1) Crowdsourcing: Multiple people doing something. New economic value.\\ 
2) It is difficult to start some kind of projects (Radio, Television, Newspaper, etc.).\\ 
3) In other similar markets (super-computer), some time (2002), the biggest supercomputer was SETTI program (were thousands of contributors supplied his limited resources).\\
4) Things are changing: Wikipedia, Apache, are FLOSS success stories.\\
5) Crowdsourcing (SETI): it is something that can be very huge, involving many people, and done via Internet, by means different from states or big companies.\\
6) Computer Market is high advanced, Computers are cheap and gives you freedom to:\\
- Do whatever you want with it.\\
- Share it to do other things, help others.\\
7) The most collaborative company might be Google. Google page rank algorithm takes into account what the rest of the people consumes, and what the rest of the people link. They won Open-Directory (Free Software). They take profit of what people does not value, or what people want to do.\\
There are cases as Google Goggles or Scanning words projects in Captchas.\\
\\
\begin{tabular}{| l | c | r |}
   \hline
   & Market Based & Non-Market Based \\
   \hline
   Decentralized  & Price Based    & \textbf{Social Sharing/Exchange} \\
   \hline
   Centralized    & Firm Hierarchy & Government/non profits (NGOs) \\
   \hline
\end{tabular}
\\
\\
(*1)In past time this paradigm was used for small projects (look after of a garden in a neighbourhood)\\
It is normally valid for "non-material" things, changing the sense of property. That happens on software, music, etc.\\
Non-market decentralized models need new means of coordination.\\
Means competition: (FLOSS vs Microsoft, Skype vs Telecomms, Wikipedia vs Encyclopedias, etc.)
The results of the competition means also the kind of the society we will live.

\section{Part II}\label{Part II} % TODO: Change for properly section title.

\section{Conclusions}\label{conclusions}
 
