\chapter{Libre software and open innovation}

\section{Introduction}

\section{Part I}\label{Part I}  % TODO: Change for properly section title.


\section{Part II}\label{Part II} % TODO: Change for properly section title.


 
% open innovation and open source 
% done by Amal Roumi

\section* {Open source and Open innovation} \label{Open source and Open innovation}

“Open innovation is a paradigm that assumes that firms can and should use external ideas as well as internal ideas, and internal and external paths to market, as the firms look to advance their technology”
‘Open innovation’ is a term coined by Professor of Business Henry Chesbrough in his 2003 book Open Innovation: The New Imperative for Creating and Profiting from Technology. 

In the years since its publication, Chesbrough’s ideas on how technology should be managed and exploited have become extremely influential. Over the same period, the public profile of free and open source software (FOSS) has risen. 
So we can say that FOSS is an example of open innovation in software.
They have two main element: use the technology, and collaborative development of that technology, unlike many individual participants, companies must also consider an economic return to justify there investment in.


\subsection* {Patterns of open innovation in free software} \label{Patterns of open innovation in free software}
There are four patterns of open innovation in free software:
\begin{itemize}
 \item  \textbf {Pooled RnD}: Pool resources to innovate in a common platform, exploit results.
 \begin{itemize}
		\item Maximization: concentrate in their own needs.
		\item Incorporation: shared technology in their products.
		\item Motivation: pool of contributors assumed.
	\end{itemize}
Lunix, Mozilla For both donate RnD to open source project while exploiting the pooles RnD of all contributors to facilitate the sale of related products.


 \item  \textbf {Spintouts}: Release the poetical of technologies within the firm that are not creating value.and they have opportunities to release more value from their technologies by position them out the firm but at the same time maintaing an ongoing corporate involvement. Here the “spinout”come.
	\begin{itemize}
		\item Maximization of impact of non-core technologies.
		\item Incorporation of contributions by third parties.
		\item Motivation: self-sustainable (or less resource-consuming) communities.
	\end{itemize}
Eclipse, Beehive,Jikes projects the sponsor firms spinout open source project that were closely aligned with firm’s ongoing strategies

 \item \textbf{Selling complements:} Income from complements, shared innovation in a common core.
	\begin{itemize}
		\item Maximization by centering on core products.
		\item Incorporation of “free” external innovation.
		\item Motivation: self-sustainable (or less resource-consuming) communities.
	\end{itemize}
Two open source examples are the IBM’s WebShere and Apple’s Safari browser.
Android,KDE .Apache, Konqueror,,Darwin open source project the firm adopting open source components.

 \item \textbf{Donated complements:} firms make their money off of the core innovation, but seek donated labor for valuable complements. Firms have indirectly and directly supported users collaboration that is coordinated using open source technologies.

	\begin{itemize}
		\item Maximization: more value for internal innovation (coreproduct).
		\item Incorporation: complements are attracted innovation.
		\item Motivation: developers involved in the core product, but willing more functionality.
	\end{itemize}
Examples: Early BSD Unix, Matlab Central, PC Game “Mods”,Avalache .
\end{itemize}

 %  Amal Roumi

 


\section{Conclusions}\label{conclusions}
